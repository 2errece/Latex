\documentclass[11pt]{article}
\usepackage[utf8]{inputenc}
\renewcommand{\contentsname}{Índice}
\renewcommand*{\familydefault}{\sfdefault}
\begin{document}

\title{Cómo escribir un documento en Latex}
\author{Rubén Recio Cantarero}
\date{Febrero del 2015}
\maketitle

\clearpage

\tableofcontents{Índice}

\clearpage

\section{Sección 1}
\subsection{Apartado 1}

Ésta es una prueba de escritura de un documento de \LaTeX \textsc{Empezando Desde Cero}. Puede que compense en relación con hacer lo mismo con Markdown, veremos a ver.

\subsection{Apartado 2}
Éste es un segundo apartado, la estructuración en Latex es relativamente sencilla cuando se trata de documentos cortos y sencillos de redactar. La verdad es que trabajar en Vim es bastante cómodo una vez le coges la dinámica, y empiezas a ver la eficiencia de trabajar con atajos de teclado. Creo que si no es necesario, no volveré a trabajar con Word, u otros editores WYSIWIG a no ser que mi trabajo deba ser revisado por otros. No obstante, Pandoc puede hacer milagros.

Puede que incluso tenga la capacidad de llevar un control de las revisiones con GitHub para ser un killer modernillo y automáticamente intelectual!

\end{document}
